\documentclass[openright,titlepage,numbers=noenddot,headinclude,%1headlines,% letterpaper a4paper
                footinclude=true,abstractoff, % <--- obsolete, remove (todo)
                BCOR=5mm,paper=a4,fontsize=11pt,%11pt,a4paper,%
                ]{scrreprt}
                
\usepackage{graphicx}
\usepackage{listings}
\graphicspath{ {images/unitTests/} }

\begin{document}

\chapter{c unit tests}
\section{checkmk framework}
testframewerk for autocreating of c unit Tests, easy to handle
Download from : http://check.sourceforge.net/
Converts *.ts files into proper *.c test files
only change the *.ts files and no need to change .c file
usage:
	
	checkmk test.ts \textgreater test.c

for more informations look up the manual pdf


\section{.ts file}

all testfunctions have to start with \texttt{\#test}

testsuites for more clustering, but not necessary

\begin{lstlisting}
#suite test_Functions_common
	#test test1{
		int i =1;
		int j =0;
		ck_assert_msg(i==j,"message");
	}
	#test test2{
		int i =4;
		int j =0;
		ck_assert_msg(i==j,"message");
	}
	
\end{lstlisting}



\chapter{perl tests}
\section{general}
writing of the perl tests directly no such thing neccessary as checkmk
\texttt{Test::More} framework will be used
manuals: http://perldoc.perl.org/Test/More.html

\section{example}
30 tests will be made, error if this value don't fit with the amount of implemented tests

\begin{lstlisting}
use Test::More tests => 30  
use strict;
use warnings;

is(1,1); # error if 1 is not 1
ok(isCorrect); # error if isCorrect is not True

\end{lstlisting}

\chapter{python tests}
\section{general}
\begin{itemize}
\item texttt{unittest} framework is used.
\item https://docs.python.org/2.7/library/unittest.html
\item define testCase class and corresponsing tests
\item we also have to call the tests by the main function will call every function of the TestCase object starting with "test\_name1", and only those tests
\end{itemize}
\section{example}
\begin{lstlisting}
import unittest
if __name__ == '__main__':  # main function, calls every test starting with test_...
    unittest.main()
    
class GeneralTests(unittest.TestCase):
	def test_calc1(self):
		print("starting of first test")
		self.assertEqual(1,1)
	def test_calc2(self):
		print("test2");
		self.assertEqual(4,0);
		self.assertTrue(isCorrect)

\end{lstlisting}




\end{document}